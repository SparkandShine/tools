\documentclass[compress]{beamer}
\usetheme{Warsaw} 

\beamertemplatenavigationsymbolsempty
\setbeamertemplate{headline}{}


\title[<short version for footer>]{<long version for titlepage>}
\usepackage{lipsum}

\usepackage{verbatim}  %for multiline comment
\usepackage[utf8]{inputenc}
\usepackage{graphicx, epstopdf}
\epstopdfsetup{outdir=../figures/}
\usepackage{float}
\graphicspath{{../figures/}}
\renewcommand{\figurename}{Fig.}

\usepackage[capitalise]{cleveref}
%\usepackage[numbers]{natbib}
\usepackage{cite}

\usepackage{caption}
\setbeamertemplate{caption}[numbered]
\captionsetup{font=scriptsize,labelfont=scriptsize}


\usepackage{tabularx} %for automatic line-break in tabular
\usepackage{amsmath}

\usepackage{makecell}  %for table header, center and bold

\begin{comment}
\makeatletter
\setbeamertemplate{footline}
{
  \leavevmode%
  \hbox{%
  \begin{beamercolorbox}[wd=.4\paperwidth,ht=2.25ex,dp=1ex,center]{author in head/foot}%
    \usebeamerfont{author in head/foot}\insertsection
  \end{beamercolorbox}%
  \begin{beamercolorbox}[wd=.3\paperwidth,ht=2.25ex,dp=1ex,center]{title in head/foot}%
    \usebeamerfont{title in head/foot}\insertsubsection
  \end{beamercolorbox}%
  \begin{beamercolorbox}[wd=.3\paperwidth,ht=2.25ex,dp=1ex,right]{date in head/foot}%
    %\usebeamerfont{date in head/foot}\insertsubsubsection
    \usebeamerfont{date in head/foot}\insertshortdate{}\hspace*{2em}
    \insertframenumber{} / \inserttotalframenumber\hspace*{2ex} 
  \end{beamercolorbox}}%
  \vskip0pt%
}
\makeatother
\end{comment}


\makeatletter
\setbeamertemplate{footline}
{
  \leavevmode%
  \hbox{%
  \begin{beamercolorbox}[wd=.9\paperwidth,ht=2.25ex,dp=1ex,center]{author in head/foot}%
    \usebeamerfont{author in head/foot}%
    \insertsectionnavigationhorizontal{.9\paperwidth}{}{}%
  \end{beamercolorbox}%
  \begin{beamercolorbox}[wd=.1\paperwidth,ht=2.25ex,dp=1ex,right]{date in head/foot}%
    \usebeamerfont{date in head/foot} %\insertshortdate{}\hspace*{2em}
    \insertframenumber{} / \inserttotalframenumber\hspace*{2ex} 
  \end{beamercolorbox}}%
  \vskip0pt%
}
\makeatother

%font size
\usepackage{lipsum}
\newcommand\Fontvi{\fontsize{6}{7.2}\selectfont}
\newcommand\FontFormulars{\fontsize{5}{6.2}\selectfont}



\title{XOR Network Coding in Delay Tolerant Networks with Data Mules \\
%\footnote{In preparation for IEEE/ICCC (International Conference on Communications in China)}}
%\title{XOR Network Coding in DTNs with Data Mules
}

\author{Qiankun Su, Katia Jaffrès-Runser, Gentian Jakllari and Charly Poulliat}


\date{July 15th, 2015}

\thispagestyle{empty} 

\titlegraphic{
\vspace*{\fill}
\centering
   \includegraphics[scale=0.1]{logo_irit}\hspace*{1.0cm}~
   \includegraphics[scale=0.2]{logo_enseeiht}\hspace*{1.0cm}~
   \includegraphics[scale=0.2]{logo_ut}
\vspace*{\fill} 
}

%%%%%%%%%%%%%%%%%%%%%%%%%%%%%%%%%%%%%%%%%%%%%%%%%%%%%%%%%%%%%%%%%%%%%%%%
\begin{document}

\begin{frame}
	\titlepage
\end{frame}

\begin{frame}{Outline}
	\tableofcontents
\end{frame}


\section{Introduction}

\subsection{Scenario description}

\begin{frame}{Scenario description} 
	\begin{figure}[!t]
		\centering
		%\includegraphics[width=2.5in]{scenario-crop.pdf}
		\includegraphics[scale=0.5]{scenario-crop.pdf}
		\caption{Village Communication Networks}
		\label{f_scenario}
	\end{figure}
	
	
	\begin{itemize}
		\item Two remote villages communicate with each other via the city base station
		\item Data mule: bus1, bus2
		      \pause
		\item The city: heavy network loads
		\item Network coding            
	\end{itemize}        
\end{frame}



\section{Theoretical analysis and simulation results}

\subsection{Modelling}
\begin{frame}{Modelling}
	\begin{figure}[!t]
		\centering
		%includegraphics[scale=0.8]{scenario.png}
		%\includegraphics[scale=0.4]{modeling.png}
		\includegraphics[width=2.5in]{model-crop.pdf}
		\caption{Model of village communication networks}
		\label{f_modeling}
	\end{figure}
	
	Symbols:
	\begin{itemize}
		\item[$t_v$, $t_c$] the waiting time in remote villages and the city
		\item[$t_b$] the travelling time of bus 
		\item[$r$] the message interval
	\end{itemize}
	
	Assumption:
	\begin{itemize}
		\item the bandwidth is divided equally between the contending nodes
		\item buses synchronize well
		\item a node sends one message at most per unit time
	\end{itemize}
\end{frame}

\begin{frame}{XOR network coding}
	\begin{figure}[!t]
		\centering
		%includegraphics[scale=0.8]{scenario.png}
		%\includegraphics[scale=0.4]{xor_network_coding_benifits.png}
		\includegraphics[width=2.5in]{with_without_network_coding-crop.pdf}
		\caption{Benifits from network coding}
		\label{f_xor_network_coding_benifits}
	\end{figure}
	\begin{itemize}
		\item without/with network coding: 4 vs 3 transmissions
		\item the bandwidth is divided equally
		      \begin{itemize}
		      	\item without network coding: $N_{d\_wo} = \frac{1}{3} \cdot N$
		      	\item with network coding: $N_{d\_nc} = \frac{1}{3} \cdot 2 \cdot N$
		      	\item $N$ denotes the number of messages received from buses
		      \end{itemize}
	\end{itemize}
\end{frame}


\subsection{Analysis: $2 \rightarrow N$ villages}
\begin{frame}{Analysis on two villages}
	\Fontvi
	%\lipsum[1]
	
	\begin{itemize}
		\item $\frac{2t_b + t_v + t_c}{r} > \frac{t_c}{3} 
		      \Rightarrow r < \frac{3 \cdot (2t_b + t_v + t_c)}{t_c}$
		      , the city node is unable to drain all created messages \emph{with} network coding.
		      
		      \begin{equation}
		      	DP_{wo} 
		      	% = \frac{N_{d\_wo}}{N_c} 
		      	= \frac{\frac{t_c}{3}}{2 \cdot \frac{2t_b + t_v + t_c}{r}}  
		      	= \frac{t_c}{6 \cdot (2t_b + t_v + t_c)} \cdot r
		      \end{equation}
		      
		      \begin{equation}
		      	DP_{nc} 
		      	%= \frac{N_{d\_nc}}{N_c} 
		      	= \frac{2 \cdot \frac{t_c}{3}}{2 \cdot \frac{2t_b + t_v + t_c}{r}} 
		      	= \frac{t_c}{3 \cdot (2t_b + t_v + t_c)}\cdot r
		      \end{equation}
		      
		      \begin{equation}
		      	G(r) = DP_{nc} - DP_{wo} =  \frac{t_c}{6 \cdot (2t_b + t_v + t_c)} \cdot r 
		      \end{equation}
		      
		\item $\frac{3 \cdot (2t_b + t_v + t_c)}{t_c} \le r < \frac{4 \cdot(2t_b + t_v + t_c)}{t_c}$
		      , the city node is unable to drain all created messages \emph{without} network coding.
		      
		      \begin{equation}
		      	DP_{nc} = 1
		      \end{equation}
		      
		      \begin{equation}
		      	DP_{wo} 
		      	% = \frac{N_{d\_wo}}{N_c} 
		      	= \frac{t_c - 2 \cdot \frac{2t_b + t_v + t_c}{r}}{2 \cdot \frac{2t_b + t_v + t_c}{r}}  
		      	= \frac{t_c}{2 \cdot (2t_b + t_v + t_c)} \cdot r - 1
		      \end{equation}
		      
		      \begin{equation}
		      	G(r) = DP_{nc} - DP_{wo} = 2 - \frac{t_c}{2 \cdot (2t_b + t_v + t_c)} \cdot r
		      \end{equation}
		      
		\item $r \ge \frac{4 \cdot(2t_b + t_v + t_c)}{t_c}$
		      , the city node is capable of draining all created messages with or without network coding.
		      
		      \begin{equation}
		      	DP_{nc} = 1
		      \end{equation}
		      
		      \begin{equation}
		      	DP_{wo} = 1
		      \end{equation}
		      
	\end{itemize}
	
\end{frame}

\begin{frame}{Simulation on two villages}
	Assign $100$ to $t_v, t_b, t_c$,
	
	\begin{equation*}
		\resizebox{.5\textwidth}{!}{$\displaystyle % restart math mode!
			\left\{
			\begin{aligned}
				DP_{wo} & = \frac{1}{24} \cdot r    & DP_{nc} & =\frac{1}{12} \cdot r, & r<12          \\
				DP_{wo} & = \frac{1}{8} \cdot r - 1 & DP_{nc} & =1 ,                   & 12 \le r < 16 \\
				DP_{wo} & =1                        & DP_{nc} & = 1,                   & r \ge 16      
			\end{aligned}
			\right.
		$}
	\end{equation*}
	
	\begin{figure}[!t]
		\centering
		\includegraphics[scale=0.2]{villageUpbound_delivery_prob.png}
		%\includegraphics{villageUpbound_delivery_prob.pdf}
		\caption{The delivery probability in different message intervals ($N_v = 2$)}
		\label{f_delivery_prob}
	\end{figure}
	
\end{frame}

%%%%%%%%%%%%%%%%%%% N villages %%%%%%%%%%%%%%
\begin{frame}{Analysis on $N$ villages (1/2)}
	\Fontvi
	\begin{itemize}
		
		\item $\frac{2t_b + t_v + t_c}{r} \ge \frac{t_c}{N_v + 1} 
		      \Rightarrow r \le \frac{(N_v + 1)(2t_b + t_v + t_c)}{t_c}$
		      , no channel access is wasted.
		      
		      \begin{equation}
		      	DP_{nc} 
		      	% = \frac{N_{d\_nc}}{N_c} 
		      	= \frac{2 \cdot \frac{t_c}{N_v + 1}}
		      	{N_v \cdot \frac{2t_b + t_v + t_c}{r}} 
		      	= \frac{2 \cdot t_c \cdot r}{N_v \cdot (N_v + 1) \cdot (2t_b + t_v + t_c)}
		      \end{equation}
		      
		      \begin{equation}
		      	DP_{wo}
		      	% = \frac{N_{d\_wo}}{N_c} 
		      	= \frac{\frac{t_c}{N_v + 1}}
		      	{N_v \cdot \frac{2t_b + t_v + t_c}{r}} 
		      	= \frac{t_c \cdot r}{N_v \cdot (N_v + 1) \cdot (2t_b + t_v + t_c)}
		      \end{equation}
		      
		      \begin{equation}
		      	G(N_v, r) = DP_{nc} - DP_{wo} =  \frac{t_c \cdot r}{N_v \cdot (N_v + 1) \cdot (2t_b + t_v + t_c)}
		      \end{equation}
		      
		\item $\frac{(N_v + 1)(2t_b + t_v + t_c)}{t_c} < r < \frac{3 \cdot N_v \cdot (2t_b + t_v + t_c)}{2 \cdot t_c}$
		      , the city node is unable to drain all created messages \emph{with} network coding.
		      
		      \begin{equation}
		      	DP_{wo} 
		      	= \frac
		      	{t_c - \frac{2t_b + t_v + t_c}{r} \cdot N_v}
		      	{\frac{2t_b + t_v + t_c}{r} \cdot N_v} 
		      	= \frac{t_c \cdot r}{N_v \cdot (2t_b + t_v + t_c)} - 1
		      \end{equation}
		      
		      
		      \begin{equation}
		      	DP_{nc} 
		      	= \frac
		      	{(t_c - \frac{2t_b + t_v + t_c}{r} \cdot N_v) \cdot 2}
		      	{\frac{2t_b + t_v + t_c}{r} \cdot N_v} 
		      	= \frac{2 \cdot t_c \cdot r}{N_v \cdot (2t_b + t_v + t_c)} - 2
		      \end{equation}
		      
		      
		      \begin{equation}
		      	G(N_v, r) = DP_{nc} - DP_{wo} =  \frac{t_c \cdot r}{N_v \cdot (2t_b + t_v + t_c)} - 1
		      \end{equation}
		      
	\end{itemize}
\end{frame}

\begin{frame}{Analysis on $N$ villages (2/2)}
	\Fontvi
	\begin{itemize}
		
		\item $\frac{3 \cdot N_v \cdot (2t_b + t_v + t_c)}{2 \cdot t_c} \le r < \frac{2\cdot N_v \cdot (2t_b + t_v + t_c)}{t_c}$
		      , the city node is unable to drain all created messages \emph{without} network coding.
		      
		      \begin{equation}
		      	DP_{nc} = 1
		      \end{equation}
		      
		      \begin{equation}
		      	DP_{wo} 
		      	%= \frac{N_{d\_wo}}{N_c} 
		      	= \frac{t_c - \frac{2t_b + t_v + t_c}{r} \cdot N_v}{\frac{2t_b + t_v + t_c}{r} \cdot N_v}  
		      	= \frac{t_c \cdot r}{(2t_b + t_v + t_c) \cdot N_v} - 1
		      \end{equation}
		      
		      \begin{equation}
		      	G(N_v, r) = DP_{nc} - DP_{wo} = 2 - \frac{t_c \cdot r}{(2t_b + t_v + t_c) \cdot N_v}
		      \end{equation}
		      
		\item $r \ge \frac{2\cdot N_v \cdot (2t_b + t_v + t_c)}{t_c}$
		      \begin{equation}
		      	DP_{nc} = 1
		      \end{equation}
		      
		      \begin{equation}
		      	DP_{wo} = 1
		      \end{equation}
		      
	\end{itemize}
\end{frame}

\begin{frame}{Simulation on $N$ villages}
	Assign $100$ to $t_v, t_b, t_c$, $4$ to $N_v$,
	
	\begin{equation*}
		\resizebox{.4\textwidth}{!}{$\displaystyle % restart math mode!
			\left\{
			\begin{aligned}
				DP_{wo} & =\frac{1}{80} \cdot r    & DP_{nc} & = \frac{1}{40} \cdot r,   & r \le 20      \\
				DP_{wo} & =\frac{1}{16} \cdot r -1 & DP_{nc} & = \frac{1}{8} \cdot r -2, & 20 < r < 24   \\
				DP_{wo} & =\frac{1}{16} \cdot r -1 & DP_{nc} & = 1,                      & 24 \le r < 32 \\
				DP_{wo} & =1                       & DP_{nc} & = 1,                      & r \ge 32      
			\end{aligned}
			\right.
		$}
	\end{equation*}
	
	\begin{figure}[!t]
		\centering
		\includegraphics[scale=0.2]{villageUpboundCross_delivery_prob.png}
		\caption{The delivery probability in different message intervals ($N_v = 4$)}
		\label{f_delivery_prob_cross}
	\end{figure}
	
\end{frame}


\subsection{Different overlapping intervals}
\begin{frame}{Different overlapping intervals}
	\begin{figure}
		\centering
		\includegraphics[scale=0.5]{overlapping_intervals-crop.pdf}
		\caption{Overlapping intervals}
		\label{f_overlapping_intervals}
	\end{figure}
	
	\begin{itemize}
		\item The city node could not drain messages from buses \emph{with} network coding.
		      \Fontvi
		      \begin{equation}
		      	DP_{wo} 
		      	% = \frac{N_{d\_wo}}{N_c} 
		      	= \frac
		      	{(t_c - t_{12}) + \frac{t_{12}}{3}}
		      	{2 \cdot \frac{2t_b + t_v + t_c}{r}}
		      	= \frac{3 \cdot t_c - 2 \cdot t_{12}}{6 \cdot (2t_b + t_v + t_c)} \cdot r
		      \end{equation}
		      
		      \begin{equation}
		      	DP_{nc} 
		      	%= \frac{N_{d\_nc}}{N_c} 
		      	= \frac
		      	{(t_c - t_{12}) + \frac{t_{12}}{3} \cdot 2}
		      	{2 \cdot \frac{2t_b + t_v + t_c}{r}} 
		      	= \frac{3 \cdot t_c - t_{12}}{6 \cdot (2t_b + t_v + t_c)}\cdot r
		      \end{equation}
		      
		      
		      \begin{equation}
		      	G(r) = DP_{nc} - DP_{wo} =  \frac{t_{12}}{6 \cdot (2t_b + t_v + t_c)} \cdot r 
		      \end{equation}
	\end{itemize}
	
\end{frame}

\begin{frame}{Simulation on different overlapping intervals}
	Assign $100$ to $t_v, t_b, t_c$, $12$ to $r$,
	\begin{figure}[!t]
		\centering
		%includegraphics[scale=0.8]{scenario.png}
		\includegraphics[scale=0.3]{villageUpboundDiffOverlap_delivery_prob.png}
		\caption{The delivery probability in different overlapping time}
		\label{f_diff_overlapping_wait_time}
	\end{figure}
\end{frame}

\section{Discussion on expanding to real networks}
\begin{frame}{Expand to real networks (1/2)} 
	\begin{itemize}
		\item tisseo Toulouse open data 
		\item the overlapping time of pair of buses
		\item the amount of data that two buses can exchange one day
		      \Fontvi
		      \begin{equation}
		      	D_{wo} = [ min(t_1, t_2) + \frac{1}{3} \cdot t_{12} ] \cdot R_b
		      \end{equation}
		      
		      \begin{equation}
		      	D_{nc} = [ min(t_1, t_2) + \frac{2}{3} \cdot t_{12} ] \cdot R_b
		      \end{equation}
		      
	\end{itemize}
	
\end{frame}



\begin{frame}{Expand to real networks (2/2)} 
	\Fontvi
	Assign $R_b = 100Mb/s$,
	
	\begin{figure}
		\centering
		%includegraphics[scale=0.8]{scenario.png}
		\includegraphics[scale=0.3]{pair_buses_throughput_improvement.png}
		\caption{The throughput improvement achieved by network coding}
		\label{f_throuhgput_improvement}
	\end{figure}
\end{frame}


\section{Ongoing work}
\begin{frame}{Ongoing work}
	\begin{itemize}
		\item more realistic
	\end{itemize}
\end{frame}



%\begin{frame}[plain,c]
\begin{frame}[c]{\space}
	\begin{center}
		\Huge Thanks for your attention.
	\end{center}
	
	\centering
	qiankun.su@enseeiht.fr
	
\end{frame}



\end{document}
