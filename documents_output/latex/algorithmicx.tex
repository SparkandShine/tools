\documentclass{article}

\usepackage{algorithm}			%  float wrapper for algorithms.
% Don’t need to manually load the algorithmicx package, as this is done by algpseudocode.
\usepackage{algorithmicx}		% algorithm typesetting environment. 
\usepackage{algpseudocode}	% layout for algorithmicx

\usepackage{amsmath}			% AMS mathematical facilities for LATEX

% Nice looking for empty set
\usepackage{amssymb}			%  provides an extended symbol collection
\let\oldemptyset\emptyset
\let\emptyset\varnothing

% Change the name of the label Algorithm
%\makeatletter
%\renewcommand{\ALG@name}{Procedure}
%\makeatother

% Rename require/ensure to input/output:
%\renewcommand{\algorithmicrequire}{\textbf{Input:}}
%\renewcommand{\algorithmicensure}{\textbf{Output:}}

\begin{document}

% algorithm 1
\begin{algorithm}
	\begin{algorithmic}%[1] % The number tells where the line numbering should start
		\caption{Coding procedure} \label{algorithm:encoding}
		\State $B=\{\text{Buses waiting at the station } \}$
		\State $S=\{\text{Next hop nodes to be reached by $B$ }\}$
		\ForAll{$s_i \in S $}
			\ForAll{$s_j \in S, j \neq i$}
				\If{$Q_{ij} \neq \emptyset \text{ and } Q_{ji} \neq \emptyset$}
					\State $m_i$ is picked at the head of $Q_{ij}$
					\State $m_j$ is picked at the head of $Q_{ji}$
					\State \Return $m_c = m_i \oplus m_j$
				\EndIf
			\EndFor
		\EndFor				
	\end{algorithmic}
\end{algorithm}

% algorithm 2
\begin{algorithm}
	\begin{algorithmic}[1]
		\caption{CDS with betweenness centrality} \label{algorithm: cds bw}
		\Require A connected graph $G(V, E)$
		\State $d \gets \{v : bw(v)\}, v \in V$, sort by BW on ascending order
		\State $V' \gets \emptyset$, connected dominating sets
		\ForAll{$v$ : $bw(v), v \notin V'$}
        	\If{$bw(v) = 0$ OR $G(V-\{v\})$ is connected}
        	    \State $V' \gets V' \cup MAX-BW(N(v))$		
        	\Else
        	    \State $V' \gets V' \cup \{v\}$
        	\EndIf
        	\State $V \gets V-\{v\}$
		\EndFor				
	\end{algorithmic}
\end{algorithm}

% algorithm 3
\begin{algorithm}
    \caption{Euclid’s algorithm}
    \label{euclid}
    \begin{algorithmic}[1] % The number tells where the line numbering should start
        \Procedure{Euclid}{$a,b$} 				\Comment{The g.c.d. of a and b}
            \State $r\gets a \bmod b$
            \While{$r\not=0$} 						\Comment{We have the answer if r is 0}
                \State $a \gets b$
                \State $b \gets r$
                \State $r \gets a \bmod b$
            \EndWhile\label{euclidendwhile}
            \State \Return $b$						\Comment{The gcd is b}
        \EndProcedure
    \end{algorithmic}
\end{algorithm}

\end{document}